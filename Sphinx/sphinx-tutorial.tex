\documentclass[a4paper,11pt]{article}

\usepackage{a4wide}
\usepackage{microtype}
\usepackage[utf8]{inputenc}
\usepackage[ngerman]{babel}
\usepackage[default,light,semibold]{sourceserifpro}
\usepackage[T1]{fontenc}
\usepackage{mathtools}
\usepackage[dvipsnames]{xcolor}
\usepackage[marginal, norule, perpage]{footmisc}
\usepackage{tabularx}
\usepackage{graphicx}
\usepackage{hyperref}
\usepackage{minted}

%settings
\usemintedstyle{lovelace}
\hypersetup{
  colorlinks=true,
  linkcolor=MidnightBlue,
  urlcolor=MidnightBlue
}

\renewcommand{\thefootnote}{\Roman{footnote}}
\def\arraystretch{1.5}

%custom commands
\newcommand{\lskip}{\vspace{0.5 em} \\}

%preamble
\title{
    \huge{Sphinx} \vspace{0.5 em} \\
    \large{Ein einfacher Einstieg}
}

\author{Markus Reichl}

\begin{document}

\maketitle
%\tableofcontents
% \begin{thebibliography}{9}
% \end{thebibliography}

\subsection*{Quickstart}
\begin{minted}{python}
    # Install
    $ pip install Sphinx
    # Quickstart
    $ sphinx-quickstart
    # Apidoc
    $ sphinx-apidoc -o ./docs/source ./package
    # Build
    $ sphinx-build -b html ./docs/source ./docs/build
    # Make
    $ make html
\end{minted}

\section*{Erweitert}
Sphinx ist ein Framework zur Dokumentation von Python Projekten.
Zur Veranschaulichung wird hier das Projekt \texttt{PyCake} verwendet, welches folgende Struktur aufweist.
\begin{minted}{python}
    PyCake/ # Aktuelles Verzeichnis
        docs/   # Zielverzeichnis der Dokumentation
        control/
            control.py
        view/
            view.py
        main.py
\end{minted}
Ziel ist es alle Klassen in diesem Projekt zu dokumentieren und als HTML Seite zur Verfügung zu stellen.

\newpage
\section{Installation}
Die aktuellste Version von Sphinx kann einfach über pip installiert werden.\footnote{Auf einigen Linux Systemen sind Administrator Rechte notwendig, dazu \mintinline{python}{sudo -H pip install Sphinx}}
\begin{minted}{python}
    $ pip install Sphinx
\end{minted}

\section{Setup}
Zur Einrichtung von Sphinx für das Projekt
\begin{minted}{python}
    $ sphinx-quickstart
\end{minted}
Hier werden einige Fragen gestellt, welche zur Erstellung der Konfigurationsdatei \texttt{conf.py} dienen.
\lskip{}
Es wird empfohlen folgende Fragen mit Ja (\texttt{y}) zu beantworten.
\begin{verbatim}
    Separate source and build directories
    autodoc: automatically insert docstrings from modules
    viewcode: include links to the source code of documented Python objects
    Create Makefile?
\end{verbatim}

\section{Dokumentation}
\subsection{Hinzufügen von Modulen}
Bevor das Projekt dokumentiert werden kann müssen für die einzelnen Module Dateien generiert werden.
Diese können manuell oder mittels
\begin{minted}{python}
    $ sphinx-apidoc -o outputdir packagedir
\end{minted}
eingerichtet werden. In unseren Fall würde dieser Ablauf wie folgt aussehen.
\begin{minted}{python}
    $ sphinx-apidoc -o ./docs/source .            # Aktuelles Paket -> main.py
    $ sphinx-apidoc -o ./docs/control ./control   # Control -> control.py
    $ sphinx-apidoc -o ./docs/view ./view         # View -> view.py
\end{minted}
Zusätzlich muss innerhalb der generierten Module der Name des Pakets vor der Modulbezeichnung definiert sein.
So wird zum Beispiel in der Datei \texttt{control.rst} die Zeile \\
\texttt{.. automodule:: control} mit \texttt{.. automodule:: control.control} ersetzt.

\subsection{Generierung}
Zuletzt kann die Dokumentation mittels
\begin{minted}{python}
    $ sphinx-build -b html ./docs/source ./docs/build
    $ make html
\end{minted}
als HTML generiert werden.

\end{document}
